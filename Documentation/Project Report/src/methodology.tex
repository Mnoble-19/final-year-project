\section{Introduction}
This chapter discusses the techniques that were used to achieve the objective of the proposal. It covers the approaches as well as techniques used for data collection, design, final implementation, and system testing.


\section{Data Gathering and Elicitation}
The team collected relevant and appropriate data to determine the requirements of the proposed system. This was done through questionnaires, interviews and physical observation\cite{kothari_research_2004}.

\subsection{Questionnaires}
Questionnaires contain open or closed ended questions given to a selection respondents to solicit information on a selected research topic\cite{bartram_using_2019}. Google forms were be used to create online questionnaires that will be shared through social media and emails to targeted stakeholders. Advantages of online questionnaires are that they are relatively easy and quick to distribute. It is also quicker to receive responses and the data can be collected directly into the database for analysis.

\subsection{Interviews}
An interview is a one on one planned conversation with a person with an aim of attaining information. Researchers had interactions with owners of the system currently in place as well as a selection of the motorists in order to get firsthand information on issues faced. This helped us in better defining the system requirements of the project. The interviews also enable the researchers to establish relationship with potential respondents as well as owners of the current system and therefore gain their corporation,  yielding highest response rates in the survey research.

\subsection{Observation}
Observation involve spending time with stakeholders and keenly monitoring their activities. This will provide unbiased information that will benefit the study. Researchers will take a physically closer look at what takes place during the daily activities of motorists trying to access the university and the challenges they face and will entail systematic noting and monitoring of events, behaviors of the motorists as they go on with their activities


\section{Data Analysis}
This will be done to remove inconsistencies in the data collected, as well as sieve out useful data that will be used to improve the system requirements.


\section{System Design}
For the system design, context diagrams will be created and used to define the scope of the project and its environment as well as the entities who will interact with it.Additionally, there will be detailed processes and data modelling\cite{rumbaugh_object-oriented_1991}.

\subsection{Process Modelling}
Here, data flow diagrams will be used to demonstrate the processes and entities that interact with the system.

\subsection{Data Modelling}
Entity relationship modelling will also be carried out to identify the entities as well as their relationship how they interact with each other
This will be achieved by use modelling entity relationships. The researchers will use this top down approach to identify the entities interacting with the system and relationship between the data that must be represented in the model.


\section{System Implementation}
At this stage the team will of the E-Tolls application.A number of tools and technologies will be used and these are defined below:

\subsection{Software tools}
\begin{itemize}
    \item  Android Studio.
    \item Arduino IDE: This will be used to write the code for the microcontroller.
    \item Flutter mobile framework: This will be used to write the cross-platform code for the application
    \item DigitalOcean: This will be used to host the remote server.
\end{itemize}

\subsection{Hardware tools}
\begin{itemize}
    \item Microcontroller
    \item Camera to be used to scan QR codes
    \item Wi-FI GSM module to enable internet connection of the microcontroller
\end{itemize}


\section{System Testing and Validation}
Here,  the system is tested with the intention of evaluating its functionality. \cite{klaus_requirements_nodate}
\par
Prototype builds for user testing and validation will be created. These will be shared using Firebase App Distribution\cite{google_firebase_2020}. This will be done to verify certain aspects of the application such as exception handling, security as the application involves connecting to users' mobile money accounts.