\section{Introduction}
This chapter discusses the techniques that were used to achieve the objective of the proposal. It covers the approaches used for data collection, design, final implementation, and system testing.


\section{Data Gathering and Elicitation}
The team collected relevant and appropriate data to determine the requirements of the proposed system. This was done through questionnaires, interviews and physical observation\cite{kothari_research_2004}.

\subsection{Questionnaires}
Questionnaires contain open or closed ended questions given to a selection respondents to solicit information on a selected research topic\cite{bartram_using_2019}. Google forms were be used to create online questionnaires that were shared through social media and emails to targeted stakeholders. Advantages of online questionnaires are that they are relatively easy and quick to distribute. It is also quicker to receive responses and the data can be collected directly into the database for analysis.

\subsection{Interviews}
An interview is a one on one planned conversation with a person with an aim of attaining information. Researchers had interactions with owners of the system currently in place as well as a selection of the motorists in order to get firsthand information on issues faced. This helped us in better defining the system requirements of the project. The interviews also enabled the researchers to establish relationship with potential respondents as well as owners of the current system and therefore gain their corporation,  yielding highest response rates in the survey research.

\subsection{Observation}
Observation involves spending time with stakeholders and keenly monitoring their activities. This will provide unbiased information that will benefit the study. Researchers took a physically closer look at what takes place during the daily activities of motorists trying to access the university and the challenges they face and will entail systematic noting and monitoring of events, behaviors of the motorists as they go on with their activities


\section{Data Analysis}
This was done to remove inconsistencies in the data collected, as well as sieve out useful data that will be used to improve the system requirements.


\section{System Design}
For the system design, context diagrams were created and used to define the scope of the project and its environment as well as the entities who will interact with it.Additionally, there were detailed processes and data modelling\cite{rumbaugh_object-oriented_1991}.

\subsection{Process Modelling}
Here, data flow diagrams were used to demonstrate the processes and entities that interact with the system.

\subsection{Data Modelling}
Entity relationship modelling was done to identify the entities as well as their relationship how they interact with each other
This was achieved by use modelling entity relationships. The researchers used a top down approach to identify the entities interacting with the system.


\section{System Implementation}
At this stage the team built the E-Tolls application.A number of tools and technologies were used and these are defined below:

\subsection{Software tools}
\begin{itemize}
    \item  Android Studio.
    \item Arduino IDE: This will be used to write the code for the microcontroller.
    \item Flutter mobile framework: This will be used to write the cross-platform code for the application
    \item DigitalOcean: This will be used to host the remote server.
    \item
\end{itemize}

\subsection{Hardware tools}
\begin{itemize}
    \item ESP32 microcontroller
    \item Servo motor to demonstrate the opening and closing of the gate
    \item Camera to be used to scan QR codes
    \item Wi-FI GSM module to enable internet connection of the microcontroller
\end{itemize}


\section{System Testing and Validation}
Here,  the system is deployed and executed to assess its functionality using carefully planned testing strategies. The system was deployed on a server and released as a prototype for user testing and validation. The purpose of validation was to ensure that the system performs precisely as intended in a consistent and efficient manner.

To validate the system, various methods were employed, including testing the prototype with invalid data and assessing how it handles exceptions. In cases where the prototype failed to handle exceptions satisfactorily, security measures were implemented to address those scenarios. Comprehensive and systematic testing was conducted to identify any flaws in the system and the underlying database structure. The identified faults were documented, and the process was repeated until the system demonstrated compliance with the specified requirements and performance criteria.


\section{Conclusion}
In conclusion, this chapter has outlined the procedures followed in the creation of the E-Tolls Mobile Application. The system we believe will be able to meet not just the requirements of the users but also the objectives of the project.