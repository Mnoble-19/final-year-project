Access to parking places is a common challenge in many ueban areas such as universities, shopping malls, hospitals, where there is a high demand for parking spaces and a limited supply.
Makerere University in collaboratio with KSA in 2014 put in a place an utomated palrking system that wllows motorists to purchase tickets and make payemtns at any of the payment points wihtin the university premise for purposes of ensuring the smooth flow of traffic, reducing congestion, and generating revenue for the university. However, the current system of ticketing and payment for parking at Makerere University in Kampala, Uganda, suffers from several problems, such as frequent malfunction of the ticketing machines, fraud by the system employees, difficulty in locating a payment point for motorists, and hefty fines for lost tickets. These problems result in inconvenience, inefficiency, and revenue loss for both the system’s managers and motorists.

To address these problems, we propose a novel system that consists of a mobile app that enables cashless digital prepayment of the parking fees using local popular payment platforms such as MTN mobile money, Airtel Money, etc. The proposed app interacts with an embedded microcontroller at the gate that scans the QR codes and grants access. Our system aims to provide a more convenient, efficient, and transparent way of managing toll payments in the university context.

Our work is also motivated by a  growing trend of using mobile apps for parking payment in some  cities around the world. Mobile apps offer several advantages over traditional methods of parking payment, such as convenience, flexibility, security, and cost-effectiveness, amongst others. A report by Grand View Research indicates that the global mobile parking app market size was valued at USD 6.4 billion in 2020 and is expected to grow at a compound annual growth rate (CAGR) of 22.1\% from 2021 to 2028. Some examples of mobile parking apps include ParkMobile developed and used in the United States and Flowbird developed in France . Most of these apps are designed for developed countries with advanced infrastructure and technology. There is a need for developing context-specific solutions that cater to the needs and challenges of developing countries like Uganda.
\clearpage